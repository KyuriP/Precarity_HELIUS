% Options for packages loaded elsewhere
\PassOptionsToPackage{unicode}{hyperref}
\PassOptionsToPackage{hyphens}{url}
\PassOptionsToPackage{dvipsnames,svgnames,x11names}{xcolor}
%
\documentclass[
]{article}

\usepackage{amsmath,amssymb}
\usepackage{iftex}
\ifPDFTeX
  \usepackage[T1]{fontenc}
  \usepackage[utf8]{inputenc}
  \usepackage{textcomp} % provide euro and other symbols
\else % if luatex or xetex
  \usepackage{unicode-math}
  \defaultfontfeatures{Scale=MatchLowercase}
  \defaultfontfeatures[\rmfamily]{Ligatures=TeX,Scale=1}
\fi
\usepackage[]{palatino}
\ifPDFTeX\else  
    % xetex/luatex font selection
\fi
% Use upquote if available, for straight quotes in verbatim environments
\IfFileExists{upquote.sty}{\usepackage{upquote}}{}
\IfFileExists{microtype.sty}{% use microtype if available
  \usepackage[]{microtype}
  \UseMicrotypeSet[protrusion]{basicmath} % disable protrusion for tt fonts
}{}
\makeatletter
\@ifundefined{KOMAClassName}{% if non-KOMA class
  \IfFileExists{parskip.sty}{%
    \usepackage{parskip}
  }{% else
    \setlength{\parindent}{0pt}
    \setlength{\parskip}{6pt plus 2pt minus 1pt}}
}{% if KOMA class
  \KOMAoptions{parskip=half}}
\makeatother
\usepackage{xcolor}
\setlength{\emergencystretch}{3em} % prevent overfull lines
\setcounter{secnumdepth}{5}
% Make \paragraph and \subparagraph free-standing
\makeatletter
\ifx\paragraph\undefined\else
  \let\oldparagraph\paragraph
  \renewcommand{\paragraph}{
    \@ifstar
      \xxxParagraphStar
      \xxxParagraphNoStar
  }
  \newcommand{\xxxParagraphStar}[1]{\oldparagraph*{#1}\mbox{}}
  \newcommand{\xxxParagraphNoStar}[1]{\oldparagraph{#1}\mbox{}}
\fi
\ifx\subparagraph\undefined\else
  \let\oldsubparagraph\subparagraph
  \renewcommand{\subparagraph}{
    \@ifstar
      \xxxSubParagraphStar
      \xxxSubParagraphNoStar
  }
  \newcommand{\xxxSubParagraphStar}[1]{\oldsubparagraph*{#1}\mbox{}}
  \newcommand{\xxxSubParagraphNoStar}[1]{\oldsubparagraph{#1}\mbox{}}
\fi
\makeatother


\providecommand{\tightlist}{%
  \setlength{\itemsep}{0pt}\setlength{\parskip}{0pt}}\usepackage{longtable,booktabs,array}
\usepackage{calc} % for calculating minipage widths
% Correct order of tables after \paragraph or \subparagraph
\usepackage{etoolbox}
\makeatletter
\patchcmd\longtable{\par}{\if@noskipsec\mbox{}\fi\par}{}{}
\makeatother
% Allow footnotes in longtable head/foot
\IfFileExists{footnotehyper.sty}{\usepackage{footnotehyper}}{\usepackage{footnote}}
\makesavenoteenv{longtable}
\usepackage{graphicx}
\makeatletter
\def\maxwidth{\ifdim\Gin@nat@width>\linewidth\linewidth\else\Gin@nat@width\fi}
\def\maxheight{\ifdim\Gin@nat@height>\textheight\textheight\else\Gin@nat@height\fi}
\makeatother
% Scale images if necessary, so that they will not overflow the page
% margins by default, and it is still possible to overwrite the defaults
% using explicit options in \includegraphics[width, height, ...]{}
\setkeys{Gin}{width=\maxwidth,height=\maxheight,keepaspectratio}
% Set default figure placement to htbp
\makeatletter
\def\fps@figure{htbp}
\makeatother
% definitions for citeproc citations
\NewDocumentCommand\citeproctext{}{}
\NewDocumentCommand\citeproc{mm}{%
  \begingroup\def\citeproctext{#2}\cite{#1}\endgroup}
\makeatletter
 % allow citations to break across lines
 \let\@cite@ofmt\@firstofone
 % avoid brackets around text for \cite:
 \def\@biblabel#1{}
 \def\@cite#1#2{{#1\if@tempswa , #2\fi}}
\makeatother
\newlength{\cslhangindent}
\setlength{\cslhangindent}{1.5em}
\newlength{\csllabelwidth}
\setlength{\csllabelwidth}{3em}
\newenvironment{CSLReferences}[2] % #1 hanging-indent, #2 entry-spacing
 {\begin{list}{}{%
  \setlength{\itemindent}{0pt}
  \setlength{\leftmargin}{0pt}
  \setlength{\parsep}{0pt}
  % turn on hanging indent if param 1 is 1
  \ifodd #1
   \setlength{\leftmargin}{\cslhangindent}
   \setlength{\itemindent}{-1\cslhangindent}
  \fi
  % set entry spacing
  \setlength{\itemsep}{#2\baselineskip}}}
 {\end{list}}
\usepackage{calc}
\newcommand{\CSLBlock}[1]{\hfill\break\parbox[t]{\linewidth}{\strut\ignorespaces#1\strut}}
\newcommand{\CSLLeftMargin}[1]{\parbox[t]{\csllabelwidth}{\strut#1\strut}}
\newcommand{\CSLRightInline}[1]{\parbox[t]{\linewidth - \csllabelwidth}{\strut#1\strut}}
\newcommand{\CSLIndent}[1]{\hspace{\cslhangindent}#1}

\usepackage[noblocks]{authblk}
\renewcommand*{\Authsep}{, }
\renewcommand*{\Authand}{, }
\renewcommand*{\Authands}{, }
\renewcommand\Affilfont{\small}
\usepackage{amssymb}
\usepackage{booktabs} % Add to your preamble for cleaner table lines
\usepackage{makecell} % Add to your preamble for multi-line cells
\makeatletter
\@ifpackageloaded{caption}{}{\usepackage{caption}}
\AtBeginDocument{%
\ifdefined\contentsname
  \renewcommand*\contentsname{Table of contents}
\else
  \newcommand\contentsname{Table of contents}
\fi
\ifdefined\listfigurename
  \renewcommand*\listfigurename{List of Figures}
\else
  \newcommand\listfigurename{List of Figures}
\fi
\ifdefined\listtablename
  \renewcommand*\listtablename{List of Tables}
\else
  \newcommand\listtablename{List of Tables}
\fi
\ifdefined\figurename
  \renewcommand*\figurename{Figure}
\else
  \newcommand\figurename{Figure}
\fi
\ifdefined\tablename
  \renewcommand*\tablename{Table}
\else
  \newcommand\tablename{Table}
\fi
}
\@ifpackageloaded{float}{}{\usepackage{float}}
\floatstyle{ruled}
\@ifundefined{c@chapter}{\newfloat{codelisting}{h}{lop}}{\newfloat{codelisting}{h}{lop}[chapter]}
\floatname{codelisting}{Listing}
\newcommand*\listoflistings{\listof{codelisting}{List of Listings}}
\makeatother
\makeatletter
\makeatother
\makeatletter
\@ifpackageloaded{caption}{}{\usepackage{caption}}
\@ifpackageloaded{subcaption}{}{\usepackage{subcaption}}
\makeatother

\ifLuaTeX
  \usepackage{selnolig}  % disable illegal ligatures
\fi
\usepackage{bookmark}

\IfFileExists{xurl.sty}{\usepackage{xurl}}{} % add URL line breaks if available
\urlstyle{same} % disable monospaced font for URLs
\hypersetup{
  pdftitle={Causal Discovery on Precarity and Depression},
  pdfauthor={Kyuri Park; Leonie K. Elsenburg; Mary Nicolao; Karien Stronks; Vítor V. Vasconcelos},
  colorlinks=true,
  linkcolor={blue},
  filecolor={Maroon},
  citecolor={Blue},
  urlcolor={Blue},
  pdfcreator={LaTeX via pandoc}}


\title{Causal Discovery on Precarity and Depression}


\author[1]{Kyuri Park}
\author[2]{Leonie K. Elsenburg}
\author[2]{Mary Nicolao}
\author[2]{Karien Stronks}
\author[1, 3]{Vítor V. Vasconcelos}

\affil[1]{\textit{Computational Science Lab, Informatics Institute, University of Amsterdam, PO Box 94323, Amsterdam, 1090GH, the Netherlands}}
\affil[2]{\textit{Department of Public and Occupational Health, Amsterdam Public Health Research Institute, Amsterdam UMC, University of Amsterdam, Amsterdam, the Netherland}}
\affil[3]{\textit{Institute for Advanced Study, University of Amsterdam, Oude Turfmarkt 147, Amsterdam, 1012GC, the Netherland}}


\date{2024-12-23}
\begin{document}
\maketitle
\begin{abstract}
\noindent Understanding the causal mechanisms linking precarity factors
and depression is critical for developing effective interventions. This
study utilizes the HELIUS dataset to explore these relationships using
advanced causal discovery methods. By applying algorithms such as FCI,
and CCI, and combining traditional Gaussian CI tests with non-parametric
approaches like RCoT, we investigate how precarity factors---including
employment, social, financial, housing, and relational stress---affect
depression, both as a sum score and at the individual symptom level. Our
findings reveal that relational stress consistently emerges as a
potential causal factor for depression, while symptoms such as sleep
disturbances, guilt, and anhedonia are particularly sensitive to
external stressors, acting as potential early warning signals or
intervention points for prevention. Moreover, the results highlight
complexities in the data, including the influence of latent confounders
and the challenges of capturing cyclic relationships. Despite some
limitations, such as unresolved ambiguities in causal directions and
challenges with mixed data distributions, this study demonstrates the
utility of causal discovery tools in disentangling the intricate
interplay between social and mental health dynamics. By mapping these
causal structures into computational models, future research can
simulate intervention effects, providing actionable insights to mitigate
the impact of precarity on mental health. This study serves as a
foundational effort, offering both methodological advancements and
practical implications for addressing depression at a population level.
\end{abstract}

\renewcommand*\contentsname{Table of contents}
{
\hypersetup{linkcolor=}
\setcounter{tocdepth}{3}
\tableofcontents
}

\section{Introduction}\label{introduction}

Mental health problems in urban areas have been reported to be on the
rise. Governments have been attempting to intervene, but the complexity
of mental health systems presents significant challenges in planning
effective interventions, let alone understanding the underlying
mechanisms driving these issues.

Recent research has aimed to identify underlying factors contributing to
mental health problems, often referred to as precariousness factors.
These factors encompass various aspects of life, such as employment,
social connections, financial stability, housing, and cultural
dimensions. This comprehensive perspective on precarity helps to
highlight how different aspects of life may be interconnected. While
this research has advanced our understanding of the complex interplay
between precariousness factors, a key question remains unanswered: how
do these factors influence mental health? Specifically, the lack of
directional information --- knowing what influences what --- limits our
ability to identify and prioritize effective intervention targets.

This study aims to investigate the causal relationships between
precariousness factors and mental health outcomes, with a specific focus
on depression, the most prevalent mental health issue in urban
populations. Using causal discovery methods, we explore how different
aspects of precarity influence depression and delve deeper into the
dynamics at the symptom level. By examining individual depressive
symptoms, we aim to identify which symptoms may act as initiators by
being particularly sensitive to specific precariousness factors. Through
this analysis, our goal is to uncover the causal mechanisms underlying
mental health challenges and provide a foundation for developing more
effective and targeted interventions.

\section{Methods}\label{methods}

\subsection{Data}\label{data}

We use the HELIUS dataset, which captures the diverse population of
Amsterdam across various ethnicities and provides comprehensive health
and lifestyle data. To operationalize precariousness factors, we draw on
the framework outlined in previous research (i.e., Leonie's paper) and
select a set of relevant variables.

To ensure a robust representation of each precariousness factor, we
conducted various exploratory analyses to identify consistent and
meaningful factor structures. Based on these analyses, we identified
five precariousness factors, including two related to recent stressors,
each comprising multiple variables as outlined below. Detailed
information on the exploratory analyses can be found in the
\hyperref[sec-appendix]{Appendix}.

\begin{itemize}
\tightlist
\item
  Employment precariousness: \texttt{emp\_stat}, \texttt{work\_sit}.
\item
  Social precariousness: \texttt{soc\_freq}, \texttt{soc\_adq}.
\item
  Housing precariousness: \texttt{nb\_safe}, \texttt{nb\_res},
  \texttt{nb\_rent}, \texttt{cul\_rec}.
\item
  Recent relational stressors: \texttt{frd\_brk12}, \texttt{conf12}.
\item
  Recent financial stressors: \texttt{fincri12}, \texttt{inc\_diff}.
\end{itemize}

After preprocessing, the HELIUS dataset comprises 21,628 samples. In
addition to the five precariousness factors, we include PHQ-9 scores ---
both the total sum score and individual symptom scores --- to represent
depression. In the subsequent causal discovery analysis, we will examine
depression both as an aggregated sum score and through its individual
symptom-level representations. Refer to Figure~\ref{fig-dist} for the
overall distributions of the variables used in the analysis.

\begin{figure}

\centering{

\includegraphics[width=0.9\textwidth,height=\textheight]{draft_v1_files/figure-pdf/fig-dist-1.pdf}

}

\caption{\label{fig-dist}Distributions of variables with density
overlay. \emph{P.emp} = employment precariousness; \emph{P.hou} =
housing precariousness; \emph{P.soc} = social precariousness;
\emph{S.fin} = recent financial stressors; \emph{S.rel} = recent
relational stressors; \emph{PHQsum} = PHQ-9 sum score; \emph{anh} =
anhedonia; \emph{app} = appetite; \emph{con} = concentration; \emph{dep}
= depressed mood; \emph{ene} = energy; \emph{glt} = guilty; \emph{mot} =
motor; \emph{sui} = suicidal}

\end{figure}%

\subsection{Causal Discovery}\label{causal-discovery}

There are numerous causal discovery algorithms available; however, in
this study, we focus on algorithms suited to the potential cyclic
relationships within our system. Specifically, we use FCI (Fast Causal
Inference) and CCI (Cyclic Causal Inference), both capable of accounting
for such cycles under certain conditions (Mooij \& Claassen, 2020;
Strobl, 2019). Additionally, we include the PC algorithm as a reference,
given its simplicity and prominence as one of the most widely known
causal discovery methods (Spirtes et al., 2001). For a more detailed
explanation of these algorithms, please refer to Park et al. (2024).

\begin{table}[ht]
\centering
\caption{Assumptions of causal discovery algorithms}
\begin{tabular}{lccccl}
\toprule
Algorithm & Acyclicity & Causal sufficiency & 
\makecell{Absence of \\ selection bias} & Output \\ 
\midrule
PC  & $\checkmark$ & $\checkmark$ & $\checkmark$ & CPDAG \\ 
FCI & $-^{a}$ & $\checkmark$ & $-^{a}$ & PAG \\ 
CCI & x & x & x & MAAG \\ 
\bottomrule
\end{tabular}
\caption*{\small{\textit{Note}. $^{a}$FCI can detect cycles in systems that lack selection bias and exhibit non-linear relationships (Mooij \& Claassen, 2020).}}
\end{table}

As shown in Table 1, the resulting graphs from FCI and CCI differ
slightly (\emph{PAG}: partial ancestral graph; \emph{MAAG}: maximal
almost ancestral graph) due to their reliance on different underlying
assumptions. Both encode information about causal relationships between
variables, where the presence of an edge indicates causal
\emph{ancestry}. Directed edges, \(A\) *→ \(B\), specify that \(B\) is
not an ancestor of \(A\) in every graph within the Markov equivalence
class, \(Equiv(G)\). \(A\)*---\(B\) represent cases where \(B\) is an
ancestor of \(A\) in every graph in \(Equiv(G)\). Circle endpoints,
\(A\)*-o\(B\), denote ambiguity in the ancestral relationship, meaning
that \(B\)'s ancestral status relative to \(A\) varies across graphs in
\(Equiv(G)\).

In contrast, the graph produced by the PC algorithm is a \emph{CPDAG}
(completed partially directed acyclic graph), where directed edges
(\(A\) → \(B\)) indicate that \(A\) is a direct cause (parent) of \(B\).
Unlike FCI and CCI, the CPDAG does not include circle symbols. Instead,
when the PC algorithm cannot determine the direction, it represents
uncertainty with bidirectional arrows. While PC serves as a useful
reference, its strict assumptions of acyclicity and the absence of
latent confounders limit its applicability in more complex settings.
Therefore, our primary focus remains on the results from FCI and CCI.
For completeness, all PC algorithm results are provided in the
\hyperref[sec-appendix]{Appendix}.

One practical challenge in applying these algorithms to the HELIUS
dataset is that the data does not follow a Gaussian distribution, and
the relationships among variables are unlikely to be strictly linear. To
account for this, we complement the commonly used Gaussian conditional
independence test (CI test), which relies on partial correlations, with
a non-parametric CI test based on kernel methods. However, kernel-based
non-parametric tests are computationally demanding, particularly with
large datasets like ours. To mitigate this issue, we employ Randomized
Conditional Independence Test (RCIT) and the Randomized conditional
Correlation Test (RCoT), which uses random Fourier features to
approximate the kernel methods, thereby significantly reducing the
computational cost (Strobl et al., 2019). For a more detailed
explanation of RCIT and RCoT, see Section~\ref{sec-rcot}.

\subsection{Analysis}\label{analysis}

We analyze the causal structure using two approaches: one with the PHQ
sum score representing overall depression severity, and another with
individual symptom scores. The PHQ sum score simplifies the analysis by
reducing dimensionality, which is computationally advantageous and
provides a broad, interpretable perspective on depression's relationship
with precarity factors. In contrast, individual symptom scores offer a
nuanced understanding, capturing the heterogeneous ways symptoms respond
to precarity factors. However, analyzing individual symptoms poses
methodological challenges due to their non-standard distributions and
the increased complexity introduced by the higher dimensionality. By
employing both approaches, we balance simplicity and granularity,
ensuring robustness in our findings: the PHQ sum score captures
overarching trends, while symptom-level analysis reveals detailed
dynamics, essential for tailoring specific interventions.

To evaluate the sensitivity of the results to the choice of alpha levels
and ensure consistency, we test two significance levels ---
\(\alpha = 0.01, 0.05\). To further ensure robustness, we employ
bootstrapping to generate 100 bootstrap samples For each sample, we
estimate causal graphs and retain only the edges and directions that
appear above predefined thresholds.

The analyses are conducted under the following conditions:

\begin{itemize}
\tightlist
\item
  Significance levels (\(\alpha\)): 0.01 and 0.05
\item
  Thresholds: 0.5, 0.6, 0.7, and 0.8
\item
  CI test: Gaussian CI test, RCoT
\item
  Algorithms: FCI, CCI, and PC
\end{itemize}

This setup yields 16 combinations (2 significance levels × 4 thresholds
× 2 CI tests), applied across three algorithms, with each combination
repeated for 100 bootstrap samples, resulting in a total of 1,600
resulting graphs. For analyses involving individual symptom variables,
we streamline the setup by using thresholds of 0.6 and 0.7, reducing the
number of bootstrap samples to 30.

To summarize the results, we identify the most frequently occurring edge
endpoints across different experimental setups.\footnote{The detailed
  proportion of each edge endpoint occurrence is shown in
  Section~\ref{sec-propmatrix}.} This ensures that only stable and
consistent edges are retained, providing a clearer and more reliable
understanding of the relationships between precarity factors and
depression. Refer to Figure~\ref{fig-workflow} for the analysis
workflow.

\begin{figure}

\centering{

\includegraphics[width=0.8\textwidth,height=\textheight]{img/simsetup.png}

}

\caption{\label{fig-workflow}Analysis workflow applied across all three
algorithms.}

\end{figure}%

\section{Results}\label{results}

\subsection{Depression as sum score}\label{depression-as-sum-score}

The sum score graphs provide a high-level summary of how precarity
factors collectively influence overall depression severity, focusing on
aggregated relationships. Figure~\ref{fig-sum} illustrates the causal
relationships between precarity factors (\emph{P.hou}, \emph{P.emp},
\emph{P.soc}, \emph{S.rel}, \emph{S.fin}) and the depression sum score
(\emph{PHQsum}) under two different setups: (a) using both Gaussian CI
test and RCoT, and (b) using RCoT alone. Black edges represent
consistent causal relationships identified by both FCI and CCI, while
gray edges denote inconsistent relationships that vary between the two
methods. The edge endpoints of inconsistent gray edges are marked with
circles. In graph (a), the key pathways suggest that employment
precarity (\emph{P.emp}) and social precarity (\emph{P.soc}) do not
cause depression (\emph{PHQsum}). Social precarity does not cause recent
relational stress (\emph{S.rel}) or financial stress (\emph{S.fin}), and
these stressors are likely related by a latent confounder. Additionally,
\emph{P.emp} and \emph{S.fin} are identified as non-causes of housing
precarity (\emph{P.hou}). Both FCI and CCI detect a dependency between
\emph{P.emp} and \emph{S.fin}, but they disagree on the direction of the
relationship, leaving it unclear whether \emph{S.fin} causes
\emph{P.emp} or if the connection is mediated by an unmeasured
confounder. A similar ambiguity exists in the relationships between
\emph{S.rel} and \emph{PHQsum} and between \emph{S.fin} and
\emph{PHQsum}, with the methods diverging on whether these stressors
influence depression or are linked through latent variables.

Graph (b), derived solely from the nonparametric RCoT test. Both graphs
consistently identify that employment precarity is not a cause of
housing precarity or depression, and that social precarity does not
cause depression or financial stress. The relationship between stressors
and depression remains unresolved between the two algorithms. However,
RCoT provides greater confidence that depression is not the cause of
financial stress and suggests a stronger likelihood that the financial
stress may contribute to depression or that their relationship is
mediated by a latent confounder. The role of relational stress has
become less pronounced, as the edge between \emph{P.soc} and
\emph{S.rel} is omitted, and the relationship between \emph{S.rel} and
\emph{S.fin} is now inconsistent between the algorithms.

Overall, both graphs together highlight the potential roles of recent
stressors as being closely linked with depression, either as causes or
through latent confounders. Employment and social precarity may also be
influenced by depression, either directly or through latent variables.
Lastly, housing precarity does not directly impact depression but
remains connected to employment precarity and financial stress, either
directly or through a latent confounder.

\begin{figure}

\begin{minipage}{0.50\linewidth}

\centering{

\includegraphics[width=1\textwidth,height=\textheight]{img/both_dot.png}

}

\subcaption{\label{fig-sum-1}Using both GaussianCI and RCoT}

\end{minipage}%
%
\begin{minipage}{0.50\linewidth}

\centering{

\includegraphics[width=1\textwidth,height=\textheight]{img/rcot_dot.png}

}

\subcaption{\label{fig-sum-2}Using only RCoT}

\end{minipage}%

\caption{\label{fig-sum}Resulting graphs of precarity factors and
depression sum score using FCI and CCI}

\end{figure}%

\subsection{Individual depression
symptom}\label{individual-depression-symptom}

Moving from the sum score representation to the symptom-level graph
provides a more granular perspective on the causal relationships between
precarity factors and depression. This approach highlights the
heterogeneity in how precarity factors influence individual depressive
symptoms --- \emph{slp} (sleep), \emph{ene} (energy), \emph{app}
(appetite), \emph{mot} (motor), \emph{sui} (suicidal), \emph{anh}
(anhedonia), \emph{glt} (guilt), and \emph{dep} (depressed mood). While
the sum score graph aggregates all symptoms into a single
measure---potentially obscuring nuanced relationships---the symptom
graph uncovers distinct pathways for different symptoms. In the
symptom-level graph (Figure~\ref{fig-sym}), consistent relationships are
represented by black solid edges, while areas of disagreement between
the FCI and CCI algorithms are denoted by gray dashed edges. Endpoints
marked with circles indicate differences in directional conclusions
between the algorithms. Additionally, the navy dashed edges represent
relationships unique to the graphs generated using Gaussian CI testing.

The symptom-level graph reveals a complex and interconnected structure,
far more intricate than the sum score graph. Its denser network
highlights the strong interdependence among symptoms and suggests the
presence of latent confounding influences, as indicated by numerous
bidirectional arrows.

Certain nodes in the network emerge as more \emph{causally} central,
highlighting their potential importance as intervention points. Among
the depressive symptoms, \emph{anh}, \emph{dep}, \emph{slp}, and
\emph{glt} emerge as particularly influential, with multiple outgoing
edges (o-\textgreater) to other symptoms, suggesting their roles as
potential key drivers within the symptom network. Especially, symptoms
such as \emph{glt}, \emph{slp}, and \emph{anh} are particularly
critical, potentially acting as initiator or activator nodes within the
network due to their apparent connections with precarity factors. These
symptoms appear to be especially sensitive to external stressors,
potentially manifesting early in response to such conditions and
subsequently activating other interconnected symptom nodes.

The causal structure involving precarity factors in the symptom-level
graph is largely consistent with the patterns observed in the sum score
graphs. As in the sum graphs, relational stress (\emph{S.rel}) emerges
as a potential causal factor for depression, while employment
(\emph{P.emp}) and social precarity (\emph{P.soc}) are more likely to be
influenced by depressive symptoms. The symptom-level analysis, however,
provides more specificity by pinpointing the symptoms involved in these
relationships. For instance, \emph{slp} is identified as influencing
\emph{P.emp} or potentially through a latent confounder (\emph{slp}
o-\textgreater{} \emph{P.emp}) , while \emph{glt} appears to affect
\emph{P.soc} or may be linked via a confounder (\emph{glt}
o-\textgreater{} \emph{P.soc}). Additional edges are observed in the
graph generated using Gaussian CI tests, such as \emph{slp}
\textless-\textgreater{} \emph{S.rel} and \emph{slp} o-\textgreater{}
\emph{P.emp}. As seen in the sum graph, Gaussian CI testing tends to
produce a denser graph. In this case, causal relationships involving
\emph{slp} are particularly prominent.

Some discrepancies, however, exist between the symptom-level and sum
graphs. For example, financial stress (\emph{S.fin}) does not have any
edges with depressive symptoms in the symptom-level graph, although it
maintains associations with other precarity factors. Additionally,
housing precarity (\emph{P.hou}) becomes entirely disconnected from the
rest of the network, appearing as an isolated node.

\begin{figure}

\centering{

\includegraphics[width=0.8\textwidth,height=\textheight]{img/symptom_graph_refined.png}

}

\caption{\label{fig-sym}Resulting graphs of precarity factors and
individual depression symptoms using FCI and CCI}

\end{figure}%

\section{Discussion}\label{discussion}

The present study explored the causal relationships between precarity
factors and depression by employing both sum score and symptom-level
analyses using causal discovery algorithms. The use of both PHQ sum
scores and individual symptom scores provided a comprehensive
understanding of how different aspects of precarity influence
depression, revealing nuanced pathways that may be obscured when
considering aggregate measures alone.

Our findings highlight the significant role of recent relational stress
(\emph{S.rel}) as a potential causal factor for depression, consistently
observed across both sum score and symptom-level analyses. This
consistency underscores the profound impact interpersonal relationships
can have on mental health. The symptom-level analysis further identified
specific symptoms, such as sleep disturbances (\emph{slp}), guilt
(\emph{glt}), and anhedonia (\emph{anh}), as particularly sensitive to
external precarity conditions. These symptoms emerged as potential
initiators or activators within the depressive symptom network,
suggesting that they may serve as early warning signs or valuable
targets for preventive interventions.

Employment precarity (\emph{P.emp}) and social precarity (\emph{P.soc})
were more likely to be influenced by depressive symptoms rather than
acting as direct causes. The symptom-level graph provided additional
clarity by identifying that sleep disturbances influenced employment
precarity (\emph{slp} o-\textgreater{} \emph{P.emp}) and feelings of
guilt affected social precarity (\emph{glt} o-\textgreater{}
\emph{P.soc}). This directional insight suggests that interventions
targeting specific depressive symptoms may have downstream effects on
improving certain aspects of precarity.

Interestingly, financial stress (\emph{S.fin}) did not exhibit causal
relationships with individual depressive symptoms in the symptom-level
analysis, contrasting with its apparent role in the sum score graphs.
This divergence indicates that while financial stress may influence
overall depression severity, its impact on specific symptoms may not be
significant. Also, housing precarity (\emph{P.hou}) emerged as an
isolated node in the symptom-level graph, losing all connections with
other precarity factors. This isolation suggests that housing precarity
may operate independently of the depressive symptom network or that its
effects are not captured within the scope of the measured variables.

Several limitations should be acknowledged when interpreting these
findings. The prevalence of bidirectional arrows and circle-marked
endpoints in the graphs reflects unresolved ambiguities in the causal
relationships suggested by the data. These uncertainties underscore the
need for further research, ideally incorporating datasets with a higher
proportion of symptomatic individuals. The HELIUS dataset, being
predominantly composed of asymptomatic samples, posed challenges in
identifying clear causal directions, particularly among symptom nodes.
Future studies could address these limitations by incorporating
time-series data to leverage temporal information about the
relationships between precarity factors and depressive symptoms.
Time-series data could provide time-specific insights and track how
these relationships evolve over time. For instance, methods such as
\emph{PCMCI} (Runge et al., 2019) and \emph{tsFCI} (Entner \& Hoyer,
2010), along with other time-series adaptations of causal discovery
algorithms, could help better account for temporal dependencies and
refine the analysis.

Another limitation is the lack of clear evidence for cycles within the
symptom network, despite using algorithms designed to account for cyclic
relationships. The CCI algorithm predominantly produced bidirectional
arrows, while FCI primarily generated directional arrows, yet neither
displayed patterns indicative of definitive cyclic structures.
Addressing cyclic relationships is particularly challenging with
observational datasets alone. Future research could benefit from refined
datasets that include intervention data. Methods such as \emph{LLC}
(Hyttinen et al., 2012), \emph{NODGAS-Flow} (Sethuraman et al., 2023),
and the recently developed \emph{Bicycle} algorithm (Rohbeck et al.,
2024) are specifically designed to utilize both observational and
intervention data to uncover potential cycles. Additionally, if
time-series data becomes available, corresponding methods, as described
above, could be applied to capture repetitive patterns in variable
interactions that might suggest cyclic structures.

Lastly, regarding conditional independence (CI) testing, the differences
between the graphs generated by Gaussian CI and RCoT underscore the
methodological sensitivities in detecting causal relationships. Gaussian
CI produced denser graphs, which is somewhat counterintuitive, as the
Gaussian CI's strict linearity assumption would typically result in
fewer detected relationships, not more. A possible explanation for this
discrepancy is that Gaussian CI's reliance on partial correlations may
overestimate relationships when specific non-linear dependencies exist
in the data. In contrast, RCoT, free from such assumptions, may better
capture these patterns under such conditions. However, while RCoT is
technically non-parametric, its performance can still be influenced by
the distributional characteristics of the data. Specifically, the RBF
kernel, optimized for continuous data with smooth transitions, may
struggle to capture relationships in datasets with discrete or mixed
distributions. In such cases, the distances between discrete points may
fail to convey meaningful similarity information. As a result, RCoT
might miss certain dependencies, particularly when variables in the
dataset lack smooth continuity. Future research could address these
issues by exploring a broader range of CI testing approaches. One
traditional approach to handle this is discretizing variables and use
\(G^2\) test, which may better capture dependencies in non-continuous
datasets (Dojer, 2016; Neapolitan et al., 2004). A more promising
direction, however, lies in the development of hybrid kernels tailored
for mixed datasets, effectively integrating both continuous and discrete
variables into RCoT-like methods. By systematically employing and
comparing a wider variety of CI testing techniques, researchers could
gain more robust insights and mitigate the limitations inherent in
specific approaches. This broader exploration holds the potential to
enhance the reliability of findings, particularly in datasets with
complex and heterogeneous structures.

Despite its limitations, this study marks a meaningful step toward
understanding the mechanisms linking precarity factors and depression.
By applying causal discovery methods, it moves beyond traditional
association-based analyses, providing insights that can inform more
precise and targeted interventions. While the resulting graphs are
preliminary and contain unresolved ambiguities, they offer a valuable
starting point for leveraging causal discovery tools to investigate the
causal interplay between depression and precarity factors. A promising
next step would involve integrating these causal structures into
computational models, such as the symptom dynamic model proposed by
\textbf{our comp-model paper}. By simulating intervention effects, such
models could provide more realistic insights into how targeted actions
might influence symptom networks and precarity factors over time. For
example, interventions focused on improving sleep hygiene or alleviating
guilt could be evaluated for their cascading effects on employment and
social relationships, offering actionable guidance for designing
population-level mental health strategies. As one of the early
applications of causal discovery tools to the complex dynamics of
depression and precarity factors, this study lays a foundation for
future research. We hope it inspires further refinement of these methods
and ultimately contribute to more effective solutions for alleviating
depression and improving societal well-being.

\section{References}\label{references}

\phantomsection\label{refs}
\begin{CSLReferences}{1}{0}
\bibitem[\citeproctext]{ref-dojer2016learning}
Dojer, N. (2016). Learning bayesian networks from datasets joining
continuous and discrete variables. \emph{International Journal of
Approximate Reasoning}, \emph{78}, 116--124.

\bibitem[\citeproctext]{ref-entner2010causal}
Entner, D., \& Hoyer, P. O. (2010). On causal discovery from time series
data using FCI. \emph{Probabilistic Graphical Models}, \emph{16}.

\bibitem[\citeproctext]{ref-hyttinen2012learning}
Hyttinen, A., Eberhardt, F., \& Hoyer, P. O. (2012). Learning linear
cyclic causal models with latent variables. \emph{The Journal of Machine
Learning Research}, \emph{13}(1), 3387--3439.

\bibitem[\citeproctext]{ref-lindsay2000moment}
Lindsay, B. G., Pilla, R. S., \& Basak, P. (2000). Moment-based
approximations of distributions using mixtures: Theory and applications.
\emph{Annals of the Institute of Statistical Mathematics}, \emph{52},
215--230.

\bibitem[\citeproctext]{ref-mooij2020constraint}
Mooij, J. M., \& Claassen, T. (2020). Constraint-based causal discovery
using partial ancestral graphs in the presence of cycles.
\emph{Conference on Uncertainty in Artificial Intelligence}, 1159--1168.

\bibitem[\citeproctext]{ref-neapolitan2004learning}
Neapolitan, R. E. et al. (2004). \emph{Learning bayesian networks} (Vol.
38). Pearson Prentice Hall Upper Saddle River.

\bibitem[\citeproctext]{ref-park2024discovering}
Park, K., Waldorp, L. J., \& Ryan, O. (2024). Discovering cyclic causal
models in psychological research. \emph{Advances. In/Psychology},
\emph{2}, e72425.

\bibitem[\citeproctext]{ref-rohbeck2024}
Rohbeck, M., Clarke, B., Mikulik, K., Pettet, A., Stegle, O., \&
Ueltzhöffer, K. (2024). Bicycle: Intervention-based causal discovery
with cycles. In F. Locatello \& V. Didelez (Eds.), \emph{Proceedings of
the third conference on causal learning and reasoning} (Vol. 236, pp.
209--242). PMLR.
\url{https://proceedings.mlr.press/v236/rohbeck24a.html}

\bibitem[\citeproctext]{ref-runge2019detecting}
Runge, J., Nowack, P., Kretschmer, M., Flaxman, S., \& Sejdinovic, D.
(2019). Detecting and quantifying causal associations in large nonlinear
time series datasets. \emph{Science Advances}, \emph{5}(11), eaau4996.

\bibitem[\citeproctext]{ref-sethuraman2023nodags}
Sethuraman, M. G., Lopez, R., Mohan, R., Fekri, F., Biancalani, T., \&
Hütter, J.-C. (2023). NODAGS-flow: Nonlinear cyclic causal structure
learning. \emph{International Conference on Artificial Intelligence and
Statistics}, 6371--6387.

\bibitem[\citeproctext]{ref-spirtes2001causation}
Spirtes, P., Glymour, C., \& Scheines, R. (2001). \emph{Causation,
prediction, and search}. MIT press.

\bibitem[\citeproctext]{ref-strobl2019}
Strobl, E. V. (2019). A constraint-based algorithm for causal discovery
with cycles, latent variables and selection bias. \emph{International
Journal of Data Science and Analytics}, \emph{8}(1), 33--56.
\url{https://doi.org/10.1007/s41060-018-0158-2}

\bibitem[\citeproctext]{ref-strobl2019approximate}
Strobl, E. V., Zhang, K., \& Visweswaran, S. (2019). Approximate
kernel-based conditional independence tests for fast non-parametric
causal discovery. \emph{Journal of Causal Inference}, \emph{7}(1),
20180017.

\bibitem[\citeproctext]{ref-zhang2012kernel}
Zhang, K., Peters, J., Janzing, D., \& Schölkopf, B. (2012).
Kernel-based conditional independence test and application in causal
discovery. \emph{arXiv Preprint arXiv:1202.3775}.

\end{CSLReferences}

\section{Appendix}\label{sec-appendix}

\subsection{Precariousness factors by
Leonie}\label{precariousness-factors-by-leonie}

\begin{enumerate}
\def\labelenumi{\arabic{enumi}.}
\tightlist
\item
  EMPLOYMENT PRECARIOUSNESS
\end{enumerate}

\begin{itemize}
\tightlist
\item
  \texttt{H1\_Arbeidsparticipatie}: Working status
\item
  \texttt{H1\_WerkSit}: Which work situation most applies to you?
\item
  \texttt{H1\_RecentErv8}: Experiences past 12 months: h. You were
  sacked from your job or became unemployed (\emph{reverse})
\end{itemize}

\begin{enumerate}
\def\labelenumi{\arabic{enumi}.}
\setcounter{enumi}{1}
\tightlist
\item
  FINANCIAL PRECARIOUSNESS
\end{enumerate}

\begin{itemize}
\tightlist
\item
  \texttt{H1\_InkHhMoeite}: During the past year, did you have problems
  managing your household income?
\item
  \texttt{H1\_RecentErv9}: Experiences past 12 months: i. You had a
  major financial crisis (\emph{reverse})
\end{itemize}

\begin{enumerate}
\def\labelenumi{\arabic{enumi}.}
\setcounter{enumi}{2}
\tightlist
\item
  HOUSING PRECARIOUSNESS
\end{enumerate}

\begin{itemize}
\tightlist
\item
  \texttt{veilig\_2012}: Score safety (veiligheid) in 2012
  (\emph{reverse})
\item
  \texttt{vrz\_2012}: Score level of resources (niveau voorzieningen) in
  2012 (\emph{reverse})
\item
  \texttt{P\_HUURWON}: Percentage Huurwoningen
\end{itemize}

\begin{enumerate}
\def\labelenumi{\arabic{enumi}.}
\setcounter{enumi}{3}
\tightlist
\item
  CULTURAL PRECARIOUSNESS
\end{enumerate}

\begin{itemize}
\tightlist
\item
  \texttt{H1\_Discr\_sumscore}: Perceived discrimination: sum score of 9
  items (range 9-45)
\item
  \texttt{H1\_SBSQ\_meanscore}: Health literacy: SBSQ meanscore (range
  1-5) (\emph{reverse})
\item
  \texttt{A\_BED\_RU}: Aantal bedrijfsvestigingen; cultuur, recreatie,
  overige diensten (\emph{reverse})
\end{itemize}

\begin{enumerate}
\def\labelenumi{\arabic{enumi}.}
\setcounter{enumi}{4}
\tightlist
\item
  SOCIAL PRECARIOUSNESS
\end{enumerate}

\begin{itemize}
\tightlist
\item
  \texttt{H1\_RecentErv5}: Experiences past 12 months: e. Your steady
  relationship ended (\emph{reverse})
\item
  \texttt{H1\_RecentErv6}: Experiences past 12 months: f.~A long-term
  friendship with a good friend or family member was broken off
  (\emph{reverse})
\item
  \texttt{H1\_RecentErv7}: Experiences past 12 months: g. You had a
  serious problem with a good friend or family member, or neighbour
  (\emph{reverse})
\item
  \texttt{H1\_SSQT}: SSQT (frequency of social contact): sum score of 5
  items (range 5-20) (\emph{reverse})
\item
  \texttt{H1\_SSQSa}: SSQS (adequacy of social contact): sum score of 5
  items, category 3 and 4 not combined (range 5-20) (\emph{reverse})
\end{itemize}

\begin{center}
\includegraphics{draft_v1_files/figure-pdf/unnamed-chunk-7-1.pdf}
\end{center}

\begin{itemize}
\item
  \textbf{High} Correlations: \texttt{emp\_stat} (employment status) and
  \texttt{work\_sit} (work situation) have a strong positive correlation
  of 0.82. This suggests that individuals with higher employment status
  tend to have more secure or favorable work situations.
  \texttt{soc\_freq} (social contact frequency) shows a strong positive
  correlation with \texttt{soc\_adq} (social adequacy) at 0.61. This
  indicates that individuals with more frequent social contact also tend
  to have higher perceived adequacy of social interactions.
\item
  \textbf{Moderate} Correlations: \texttt{nb\_safe} (neighborhood
  safety) and \texttt{nb\_res} (resources) have a moderate positive
  correlation of 0.39, suggesting that areas with higher safety also
  have better resources. \texttt{hea\_lit} (health literacy) has
  moderate correlations with \texttt{emp\_stat} (0.26) and
  \texttt{work\_sit} (0.25), which could mean that higher health
  literacy is associated with better employment situations.
  \texttt{frd\_brk12} (friendship breakups) and \texttt{conf12}
  (conflicts) have a notable correlation of 0.43, indicating a
  relationship between having conflicts and friendship losses.
\item
  \textbf{Low to Moderate} Correlations in Financial Precariousness:
  \texttt{inc\_dif} (income difficulties) has a moderate correlation
  with \texttt{fincri12} (financial crisis) at 0.49. This aligns with
  the expected relationship, where individuals who experience general
  income difficulties are more likely to report financial crises.
\item
  \textbf{Low} Correlations (0.1 - 0.2): Many variables, such as
  \texttt{discrim} (discrimination), \texttt{unemp12} (unemployment
  experience), and \texttt{rel\_end12} (relationship end), have low
  correlations with other variables, suggesting relatively independent
  relationships in the context of this dataset.
\end{itemize}

\subsection{Exploratory Factor Analysis
(EFA)}\label{exploratory-factor-analysis-efa}

\begin{center}
\includegraphics[width=0.55\textwidth,height=\textheight]{draft_v1_files/figure-pdf/unnamed-chunk-8-1.pdf}
\end{center}

\begin{center}
\includegraphics[width=0.7\textwidth,height=\textheight]{draft_v1_files/figure-pdf/unnamed-chunk-9-1.pdf}
\end{center}

\subsubsection{Factor Loadings (Pattern
Matrix)}\label{factor-loadings-pattern-matrix}

\begin{itemize}
\tightlist
\item
  \textbf{MR1}: High loadings on \texttt{emp\_stat} and
  \texttt{work\_sit} suggest this factor captures \emph{employment}
  precariousness.
\item
  \textbf{MR2}: Strong loadings on \texttt{soc\_freq} and
  \texttt{soc\_adq} indicate \emph{social} precariousness.
\item
  \textbf{MR3}: Key items like \texttt{frd\_brk12}, \texttt{conf12}, and
  \texttt{fincri12}, suggest recent \emph{stressful events}.
\item
  \textbf{MR4}: High loadings on \texttt{nb\_res} and \texttt{cul\_rec}
  may reflect \emph{community resources} precariousness.
\item
  \textbf{MR5}: Variables \texttt{nb\_safe} and \texttt{nb\_rent} with
  high loadings indicate \emph{housing} precariousness.
\end{itemize}

\subsubsection{Variance Explained}\label{variance-explained}

The factors cumulatively explain \textbf{38\%} of the variance, with MR1
being the most influential factor. Each factor contributes a smaller
proportion to the total variance (MR1 at 12\%, MR2 at 9\%, etc.).

\subsubsection{Factor Intercorrelations}\label{factor-intercorrelations}

Factors are moderately correlated, especially between \emph{MR1 and
MR5}, and \emph{MR2 and MR3}. This indicates that while distinct, these
factors are related---reasonable in a complex socio-economic context.

\subsubsection{Model Fit Statistics}\label{model-fit-statistics}

RMSEA (0.071) suggest an acceptable fit. Tucker Lewis Index (0.802)
suggests moderate reliability for the model.

\subsubsection{Summary}\label{summary}

The 5-factor model appears interpretable and captures distinct
dimensions of precariousness: \emph{employment, social, stressors,
community resources, and housing precariousness}. Although the overall
fit and explained variance could be stronger, these factors offer
insights into the underlying structure of the data, highlighting key
areas of precariousness.

\subsection{PCA}\label{pca}

\begin{center}
\includegraphics[width=0.7\textwidth,height=\textheight]{draft_v1_files/figure-pdf/unnamed-chunk-10-1.pdf}
\end{center}

\begin{itemize}
\tightlist
\item
  Component Retention: The scree plot shows a clear ``elbow'' after the
  first component. This steep drop suggests that most variance is
  explained by the first component. After Dimension 5, the percentage of
  explained variance decreases slightly more gradually, indicating
  diminishing returns for adding more components. If we need to choose
  multiple components, retaining the first 5 components seems
  reasonable, as they capture most of the variance (cumulatively
  explaining about 54.7\% of the total variance).
\end{itemize}

\begin{center}
\includegraphics{draft_v1_files/figure-pdf/unnamed-chunk-11-1.pdf}
\end{center}

\subsubsection{Explained variance (contributions) of
variables}\label{explained-variance-contributions-of-variables}

It shows the importance of variables within each component.

\begin{itemize}
\item
  \textbf{Dim1}: High contributions are observed from
  \texttt{emp\_stat}, \texttt{work\_sit}, \texttt{inc\_dif}, and
  \texttt{fincri12}, suggesting that this dimension captures aspects of
  \emph{employment and financial} security.
\item
  \textbf{Dim2}: While \texttt{emp\_stat} and \texttt{work\_sit} overlap
  with Dim1, the strong contributions from \texttt{frd\_brk12} and
  \texttt{rel\_end12} indicate that this dimension captures a focus on
  \emph{recent relationship stressors}.
\item
  \textbf{Dim3}: \texttt{cul\_rec}, \texttt{nb\_res} have the highest
  contributions, indicating this dimension likely represents
  \emph{community and cultural} factors.
\item
  \textbf{Dim4}: \texttt{soc\_freq} and \texttt{soc\_adq} stand out in
  this dimension, suggesting an emphasis on \emph{social}
  precariousness.
\item
  \textbf{Dim5}: \texttt{nb\_safe} and \texttt{nb\_rent} are the top
  contributors, pointing to \emph{housing} security as key themes in
  this component.
\end{itemize}

\subsubsection{Cos² Values}\label{cosuxb2-values}

Cos² (squared cosine) values, or the quality of representation, show how
well each variable is represented by each dimension. where higher cos²
values (closer to 1) indicate better representation of a variable by a
component.

\begin{center}
\includegraphics[width=0.7\textwidth,height=\textheight]{draft_v1_files/figure-pdf/unnamed-chunk-12-1.pdf}
\end{center}

\begin{itemize}
\item
  \textbf{Dim.1}: Variables \texttt{emp\_stat}, \texttt{work\_sit},
  \texttt{inc\_dif}, and \texttt{fincri12} show high cos² values,
  meaning that PC1 primarily captures variations in employment and
  financial difficulties. This component could represent
  \emph{employment \& finance} precariousness.
\item
  \textbf{Dim.2}: Variables \texttt{work\_sit}, \texttt{emp\_stat},
  \texttt{frd\_brk12}, and \texttt{conf12} are well-represented in this
  component, suggesting PC2 captures aspects of \emph{recent
  relationship stressors}.
\item
  \textbf{Dim.3}: Variables \texttt{nb\_res} and \texttt{cul\_rec} load
  strongly on PC3. This may represent community or cultural resources,
  indicating that this component is associated with \emph{neighborhood
  resources}.
\item
  \textbf{Dim.4}: This component has high cos² values for
  \texttt{nb\_res}, \texttt{cul\_rec}, \texttt{soc\_freq}, and
  \texttt{soc\_adq.} While \texttt{nb\_res} and \texttt{cul\_rec} are
  also prominent in PC3, PC4 uniquely captures nuanced differentiation
  in \emph{social} precariousness.
\item
  \textbf{Dim.5}: \texttt{nb\_safe} and \texttt{nb\_rent} are
  well-represented by PC5. This component might capture \emph{housing}
  precariousness.
\end{itemize}

\subsection{ICA}\label{ica}

\begin{center}
\includegraphics[width=0.7\textwidth,height=\textheight]{draft_v1_files/figure-pdf/unnamed-chunk-13-1.pdf}
\end{center}

\subsubsection{Dominant Variables per
Component:}\label{dominant-variables-per-component}

For each Independent Component (IC), we can identify variables with
\emph{high absolute} values in each column. These values indicate that
the IC captures a strong, independent signal associated with these
variables.

\begin{itemize}
\item
  \textbf{IC1}: \texttt{soc\_freq} and \texttt{soc\_adq} have strong
  negative loadings on this component, indicating that this component
  might represent \emph{social} precariousness.
\item
  \textbf{IC2}: \texttt{frd\_brk12}, \texttt{conf12},
  \texttt{rel\_end12}, \texttt{fincri12} and \texttt{unemp12} have the
  most substantial loadings on this component, all with negative signs.
  This might point to a \emph{recent relational or social stressor}
  component.
\item
  \textbf{IC3}: \texttt{nb\_res} and \texttt{cul\_rec} show notable
  negative loadings, pointing to a focus on \emph{community resource}
  precariousness.
\item
  \textbf{IC4}: High loadings for \texttt{nb\_safe}, \texttt{nb\_rent},
  \texttt{nb\_res}, \texttt{cul\_rec}, and \texttt{discrim} suggest a
  theme of \emph{housing and community-based} precariousness, reflecting
  both safety and social challenges within the neighborhood context.
\item
  \textbf{IC5}: \texttt{emp\_stat} and \texttt{work\_sit} both have
  strong negative loadings on this component, suggesting it captures
  \emph{employment} precariousness.
\end{itemize}

\subsection{Hierarchical clustering}\label{hierarchical-clustering}

\subsubsection{Using Euclidean distance}\label{using-euclidean-distance}

\begin{itemize}
\tightlist
\item
  Ward.D's method: Minimizes the variance within clusters, producing
  more compact and spherical clusters.
\item
  Single linkage: Groups clusters based on the minimum distance between
  points.
\item
  Complete linkage: Groups clusters based on the maximum distance
  between points.
\item
  Average linkage: Uses the average distance between all pairs of points
  in the two clusters.
\end{itemize}

\begin{figure}

\begin{minipage}{0.50\linewidth}
\begin{center}
\includegraphics{draft_v1_files/figure-pdf/unnamed-chunk-14-1.pdf}
\end{center}
\end{minipage}%
%
\begin{minipage}{0.50\linewidth}
\begin{center}
\includegraphics{draft_v1_files/figure-pdf/unnamed-chunk-14-2.pdf}
\end{center}
\end{minipage}%
\newline
\begin{minipage}{0.50\linewidth}
\begin{center}
\includegraphics{draft_v1_files/figure-pdf/unnamed-chunk-14-3.pdf}
\end{center}
\end{minipage}%
%
\begin{minipage}{0.50\linewidth}
\begin{center}
\includegraphics{draft_v1_files/figure-pdf/unnamed-chunk-14-4.pdf}
\end{center}
\end{minipage}%

\end{figure}%

\paragraph{Consistent Groupings (Across All or Most
Methods)}\label{consistent-groupings-across-all-or-most-methods}

\begin{itemize}
\item
  \texttt{emp\_stat} and \texttt{work\_sit}: This pair consistently
  clusters together across all linkage methods, suggesting that they are
  closely related variables, likely capturing a similar aspect of the
  data (possibly employment status or employment-related information).
\item
  \texttt{nb\_safe}, \texttt{nb\_res}, and \texttt{nb\_rent}: These
  variables are often grouped closely in several methods (especially
  Ward.D, average, and complete linkage). This suggests a similarity or
  common theme among them, potentially related to neighborhood or
  housing precariousness.
\item
  \texttt{soc\_freq} and \texttt{soc\_adq}: These two variables
  frequently cluster together, indicating they likely measure aspects of
  social frequency and adequacy in similar ways. They appear together in
  Ward.D, average, and complete linkage.
\item
  \texttt{frd\_brk12} and \texttt{conf12}: These variables are often
  clustered closely (though they sometimes join with other variables
  like rel\_end12), suggesting they may capture aspects of relationship
  or social conflict. This pair appears in close proximity, especially
  in average and Ward.D.
\end{itemize}

\paragraph{Inconsistent Groupings (Variability Across
Methods)}\label{inconsistent-groupings-variability-across-methods}

\begin{itemize}
\item
  \texttt{hea\_lit}: This variable shows inconsistent clustering across
  methods. In Ward.D, it joins with \texttt{fincri12}, while in other
  methods, it's often more isolated or grouped with variables that do
  not appear similar. This may suggest that \texttt{hea\_lit} does not
  strongly correlate with other variables, or it has multidimensional
  aspects affecting its grouping across methods.
\item
  \texttt{discrim}: This variable also shows variable groupings. In
  Ward.D, it is grouped with \texttt{hea\_lit}, while in other methods
  (e.g., complete and single linkage), it clusters differently,
  sometimes on its own. This variability may indicate that
  \texttt{discrim} has weaker associations with the main clusters in the
  data or overlaps partially with multiple clusters.
\item
  Social and Financial Variables (\texttt{inc\_dif}, \texttt{fincri12},
  \texttt{unemp12}): These variables appear together in some methods
  (e.g., Ward.D clusters \texttt{fincri12} and \texttt{inc\_dif}), but
  in others, they are spread out. This inconsistency suggests that
  social and financial variables may not have strong or consistent ties
  across different methods, perhaps due to capturing different aspects
  of precariousness.
\end{itemize}

\paragraph{Summary}\label{summary-1}

The consistent clusters are likely capturing distinct thematic
dimensions of the data (e.g., employment, housing, social contact),
while the inconsistent variables may reflect multifaceted or weakly
correlated attributes that do not fit neatly into one cluster.

\subsubsection{Using Mutual Information}\label{using-mutual-information}

Using mutual information (MI) as a basis for hierarchical clustering
differs from using traditional distance measures (like Euclidean
distance) in a few key ways.

\begin{center}
\includegraphics[width=0.6\textwidth,height=\textheight]{draft_v1_files/figure-pdf/unnamed-chunk-15-1.pdf}
\end{center}

\paragraph{Comparison to Euclidean Distance
Clustering}\label{comparison-to-euclidean-distance-clustering}

\begin{itemize}
\item
  \textbf{Housing and Community} Cluster: The variables
  \texttt{nb\_safe}, \texttt{nb\_res}, \texttt{nb\_rent}, and
  \texttt{cul\_rec} cluster together, indicating a strong association
  among housing-related and community-based factors. This suggests a
  shared theme of housing or community precariousness. This grouping is
  also observed in the Euclidean-based clustering, but it appears more
  tightly connected here, potentially due to the non-linear
  relationships highlighted by mutual information.
\item
  \textbf{Employment and Social Support} Cluster: \texttt{emp\_stat} and
  \texttt{work\_sit} form a cluster, linking employment status and work
  situation together as they did in Euclidean-based clustering. These
  remain closely associated regardless of the distance metric used.
  \texttt{soc\_freq} and \texttt{soc\_adq}, related to social contact
  frequency and adequacy, cluster nearby, indicating they have a
  stronger non-linear relationship with employment variables. This is a
  subtle difference as Euclidean distance might not capture this
  association as effectively.
\item
  \textbf{Financial Stressor} Cluster: \texttt{inc\_dif} and
  \texttt{fincri12}, representing income difficulties and recent
  financial crises, consistently cluster together in both approaches,
  showing a strong association, likely linear. However, mutual
  information-based clustering links these financial stressors with
  social support variables, suggesting that financial challenges may
  have complex dependencies with social support in this dataset.
\item
  \textbf{Relational Stressor} Cluster: \texttt{frd\_brk12},
  \texttt{conf12}, \texttt{discrim}, \texttt{hea\_lit},
  \texttt{unemp12}, and \texttt{rel\_end12} form a \emph{looser} cluster
  focused on social and relational stressors (e.g., friendship breakup,
  conflicts, and discrimination). Compared to Euclidean clustering,
  \texttt{discrim} and \texttt{hea\_lit} (health literacy) appear closer
  to relational stressors here, indicating that non-linear relationships
  might play a larger role in linking these variables.
\end{itemize}

\paragraph{Summary}\label{summary-2}

In conclusion, mutual information-based clustering provides an
alternative perspective that can reveal more intricate associations
between variables, especially for those with non-linear relationships.
Compared to Euclidean clustering, it shows a similar high-level
structure but emphasizes nuanced connections between variables,
particularly around social support, employment, and financial stress.

\subsection{Conclusions on Precariousness
factors}\label{conclusions-on-precariousness-factors}

Based on the consistent findings across multiple analyses, we decided to
exclude the variables \texttt{discrim}, \texttt{hea\_lit},
\texttt{umemp12}, and \texttt{rel\_end12}, as they do not clearly belong
to any specific precariousness factor nor exhibit strong associations
with depression (see the correlation table above). Therefore, we propose
retaining the following key precariousness factors:

\begin{itemize}
\tightlist
\item
  Employment Precariousness: \texttt{emp\_stat}, \texttt{work\_sit}
\item
  Social Precariousness: \texttt{soc\_freq}, \texttt{soc\_adq}
\item
  Housing Precariousness: \texttt{nb\_safe}, \texttt{nb\_res},
  \texttt{nb\_rent}, \texttt{cul\_rec}
\item
  Recent Relational Stressors: \texttt{frd\_brk12}, \texttt{conf12}
\item
  Recent Financial Stressors: \texttt{fincri12}, \texttt{inc\_diff}
\end{itemize}

We construct each precariousness factor by calculating the mean value of
the combined variables. Below, we present the updated correlation table
for the newly composed factors, along with the corresponding
distributions of all variables to be used in the causal discovery
analysis.

\begin{center}
\includegraphics{draft_v1_files/figure-pdf/unnamed-chunk-16-1.pdf}
\end{center}

\begin{center}
\includegraphics{draft_v1_files/figure-pdf/unnamed-chunk-17-1.pdf}
\end{center}

\subsection{Results from PC algorithm}\label{results-from-pc-algorithm}

\begin{figure}

\begin{minipage}{0.50\linewidth}

\centering{

\includegraphics[width=1\textwidth,height=\textheight]{img/sum_PC_all.png}

}

\subcaption{\label{fig-pc_sum-1}Using both GaussianCI and RCoT}

\end{minipage}%
%
\begin{minipage}{0.50\linewidth}

\centering{

\includegraphics[width=1\textwidth,height=\textheight]{img/sum_PC_RCoTonly.png}

}

\subcaption{\label{fig-pc_sum-2}Using only RCoT}

\end{minipage}%

\caption{\label{fig-pc_sum}Resulting graphs of precarity factors and
depression sum score using PC}

\end{figure}%

\begin{figure}

\begin{minipage}{\linewidth}

\centering{

\includegraphics[width=0.7\textwidth,height=\textheight]{img/pc_sym_GaussianCIRCoT.png}

}

\subcaption{\label{fig-pc_sym-1}Using both GaussianCI and RCoT}

\end{minipage}%
\newline
\begin{minipage}{\linewidth}

\centering{

\includegraphics[width=0.7\textwidth,height=\textheight]{img/pc_sym_onlyRCoT.png}

}

\subcaption{\label{fig-pc_sym-2}Using only RCoT}

\end{minipage}%

\caption{\label{fig-pc_sym}Resulting graphs of precarity factors and
individual depression symptoms using PC}

\end{figure}%

\clearpage

\subsection{Randomized Conditional Independence / Correlation Test (RCIT
\& RCoT)}\label{sec-rcot}

RCIT (Randomized Conditional Independence Test) and RCoT (Randomized
conditional Correlation Test) are advanced methods for scalable
conditional independence (CI) testing, offering computational efficiency
while maintaining the accuracy of kernel-based approaches. These methods
evaluate conditional independence between two variables \(X\) and \(Y\)
given a third variable \(Z\) while addressing computational challenges
inherent in kernel-based CI tests. In this section, we provide a
high-level overview of RCIT and RCoT based on (Strobl et al., 2019).

\subsubsection{Kernel-Based Conditional Independence
Testing}\label{kernel-based-conditional-independence-testing}

Traditional kernel-based CI tests, such as the Kernel Conditional
Independence Test (KCIT), compute dependencies using the Hilbert-Schmidt
Independence Criterion (HSIC) in reproducing kernel Hilbert spaces
(RKHS) (Zhang et al., 2012). KCIT uses the following hypothesis
framework: \[
H_0: X \perp\!\!\!\perp Y \,|\, Z, \quad H_1: X \not\!\perp\!\!\!\perp Y \,|\, Z.
\] The core quantity in KCIT is the partial cross-covariance operator:
\[
\Sigma_{XY \cdot Z} = \Sigma_{XY} - \Sigma_{XZ} \Sigma_{ZZ}^{-1} \Sigma_{ZY},
\] where \(\Sigma_{XY}\) represents the cross-covariance operator
between \(X\) and \(Y\), and
\(\Sigma_{XZ} \Sigma_{ZZ}^{-1} \Sigma_{ZY}\) removes the dependence
mediated by \(Z\).

The squared Hilbert-Schmidt (HS) norm of \(\Sigma_{XY \cdot Z}\) serves
as the test statistic: \[
\|\Sigma_{XY \cdot Z}\|^2_{HS} = 0 \quad \text{if and only if} \quad X \perp\!\!\!\perp Y \,|\, Z.
\]

KCIT estimates residual dependencies using kernel ridge regression: \[
f^*(z) = K_Z (K_Z + \lambda I)^{-1} f(x),
\] where \(K_Z\) is the kernel matrix for \(Z\), \(f(x)\) is the kernel
feature map for \(X\), and \(\lambda\) is the ridge regularization
parameter. The residual function for \(X\) is: \[
f_\text{res}(x) = f(x) - f^*(z) = R_Z f(x),
\] with: \[
R_Z = I - K_Z (K_Z + \lambda I)^{-1}.
\] The kernel matrix for residualized \(X\) is: \[
K_{X \cdot Z} = R_Z K_X R_Z,
\] and similarly for \(Y\), \(K_{Y \cdot Z} = R_Z K_Y R_Z\).

The test statistic is computed as: \[
T_{XY \cdot Z} = \frac{1}{n^2} \text{tr}(K_{X \cdot Z} K_{Y \cdot Z}),
\] which estimates the Hilbert-Schmidt (HS) norm of the partial
cross-covariance operator. To ensure convergence, KCIT scales the
statistic by \(n\): \[
S_K = n T_{XY \cdot Z}.
\] The null hypothesis \(H_0\) is rejected if \(S_K\) exceeds a
threshold determined by permutation or moment-matching-based null
distribution (Lindsay et al., 2000).

\subsubsection{Random Fourier Features
(RFFs)}\label{random-fourier-features-rffs}

Kernel-based methods like KCIT face scalability issues, as they involve
operations on \(n \times n\) kernel matrices, which scale quadratically
with the sample size \(n\). RCIT and RCoT overcome this bottleneck using
\emph{Random Fourier Features (RFFs)} to approximate kernel operations
efficiently.

\paragraph{Bochner's Theorem}\label{bochners-theorem}

Bochner's theorem provides the foundation for RFFs, stating that any
continuous shift-invariant kernel \(k(x, y)\) can be expressed as: \[
k(x, y) = \int_{\mathbb{R}^p} e^{i \omega^\top (x - y)} \, dP_\omega,
\] where \(P_\omega\) is the spectral distribution of the kernel. For
the widely used RBF kernel: \[
k(x, y) = \exp\left(-\frac{\|x - y\|^2}{2\sigma^2}\right),
\] \(P_\omega\) follows a Gaussian distribution:
\(\omega \sim \mathcal{N}(0, \sigma^2 I)\).

\paragraph{RFF Approximation}\label{rff-approximation}

Using Monte Carlo sampling, the kernel function is approximated as: \[
k(x, y) \approx \phi(x)^\top \phi(y),
\] where \(\phi(x)\) is the random Fourier feature mapping: \[
\phi(x) = \sqrt{\frac{2}{D}} \cos(W^\top x + b),
\] with \(W \sim \mathcal{N}(0, \sigma^2 I)\) and
\(b \sim \text{Uniform}(0, 2\pi)\). Here, \(D\) is the number of Fourier
features, which balances computational efficiency and approximation
accuracy.

\subsubsection{Differences Between RCIT and
RCoT}\label{differences-between-rcit-and-rcot}

RCIT and RCoT differ in their test statistics, computational efficiency,
and practical performance, which makes them suited for different
scenarios in causal discovery. RCIT evaluates the Hilbert-Schmidt norm
of the full partial cross-covariance operator, providing a general test
for conditional independence but at a higher computational cost. RCoT
simplifies the process by using the Frobenius norm of a
finite-dimensional residualized cross-covariance matrix, significantly
reducing complexity and improving scalability.

These distinctions are particularly important for large-scale datasets,
where RCoT's computational efficiency makes it a practical choice for
high-dimensional causal discovery tasks.

\paragraph{RCIT: Randomized Conditional Independence
Test}\label{rcit-randomized-conditional-independence-test}

RCIT tests full conditional independence by examining the squared
Hilbert-Schmidt (HS) norm of the partial cross-covariance operator
\(\Sigma_{XY \cdot Z}\): \[
S_K = n T_{XY \cdot Z} = \frac{1}{n} \text{tr}(K_{X \cdot Z} K_{Y \cdot Z}),
\] where \(T_{XY \cdot Z}\) is an empirical estimate of
\(\|\Sigma_{XY \cdot Z}\|^2_{HS}\). The null and alternative hypotheses
are: \[
H_0: \|\Sigma_{XY \cdot Z}\|^2_{HS} = 0, \quad H_1: \|\Sigma_{XY \cdot Z}\|^2_{HS} > 0.
\]

RCIT is a general test for conditional independence but becomes
computationally demanding as the size of \(Z\) increases, due to the
high-dimensional kernel operations required.

\paragraph{RCoT: Randomized Conditional Correlation
Test}\label{rcot-randomized-conditional-correlation-test}

RCoT simplifies the testing process by using a finite-dimensional
partial cross-covariance matrix, avoiding full HS norm calculations.
Instead, it uses the Frobenius norm of the residualized cross-covariance
matrix: \[
S' = n \|C_{AB \cdot C}\|_F^2,
\] where \(C_{AB \cdot C}\) represents the residualized cross-covariance
matrix. The hypotheses are: \[
H_0: \|C_{AB \cdot C}\|_F^2 = 0, \quad H_1: \|C_{AB \cdot C}\|_F^2 > 0.
\]

RCoT is computationally efficient and well-suited for large conditioning
sets (\(|Z| \geq 4\)). Its simplicity enables robust calibration of the
null distribution and improved scalability for high-dimensional data.

\subsection{Summarized Stable Edges Proportion}\label{sec-propmatrix}

\begin{figure}

\centering{

\includegraphics[width=1\textwidth,height=\textheight]{img/sumgraph_mat.pdf}

}

\caption{\label{fig-sum-mat}Proportions of edge endpoint types for
graphs based on depression sum scores}

\end{figure}%

\begin{figure}

\centering{

\includegraphics[width=1\textwidth,height=\textheight]{img/symgraph_mat.pdf}

}

\caption{\label{fig-sym-mat}Proportions of edge endpoint types for
graphs based on individual depression symptoms}

\end{figure}%




\end{document}
